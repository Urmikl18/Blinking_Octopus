\documentclass[10pt, conference, compsocconf]{IEEEtran}

\begin{document}

\title{Build your own OctopusDB: Blinktopus Edition}
\author{ \\  \\ }

\author{\IEEEauthorblockN{Pavlo Shevchenko, Ali Hashaam, Guzel Mussilova, Ali Memon}
\IEEEauthorblockA{Otto-von-Guericke-University, Magdeburg\\
firstname.lastname@st.ovgu.de}}
%%\and\IEEEauthorblockN{Gabriel Campero Durand}\IEEEauthorblockA{Otto-von-Guericke-University, Magdeburg\\gabrielcampero@acm.org}}

\maketitle

\begin{abstract}
What is this paper about?
\end{abstract}
%%\section{Abstract}

\section{Introduction}
%%Motivation. Main Idea. Goal. Structure of the paper. Contribution.
%%\subsection{Background and Motivation}
Over the last decades we are witnessing that modern enterprises need to pick only  specialized DBMSs(e.g. OLAP, OLTP, streaming systems and etc.) each tailored to their specific use-case. Consequently, it leads to additional costs in terms of licensing, maintenance, integration and man-hours for DBAs. Although, it is affordable for some companies, to adapt these integrated solutions to constantly changing workloads and requirements. However, it may still be a challenging and non-trivial task to achieve. Thus, in order to cope with these problems an implementation of a new all-purpose system could be a perfect solution. \\ Nowadays there exists a great variety of systems that claim to solve the aforementioned problems and yet their cost might be quite prohibitive. Some traditional DBMSs(e.g., Microsoft SQL Server, Oracle, ...) have already included the support of both analytical (OLAP, which is characterized by long-running queries over all the values of few columns) and transactional (OLTP, characterized by short-lived transactions that affect all attributes of few rows) workloads. Meanwhile, in the 15 years they have been observed to be not too efficient for new memory-rich architectures. Later exploiting the progress in modern hardware new DBMSs have been proposed, which have a simpler architecture than traditional disk-based ones and already proved to be better than those. Among these recent solutions are the column-stores(e.g., C-Store[13], MonetDB[8], ...) and the row-stores(e.g., Hekaton[7], H-Store[14], MemSQL[15], ...) that particularly designed for analytical and transactional processing, respectively. Still, following the \textit{one size does not fit all} observation these systems have mainly specialized either for OLAP or for OLTP[6]. Thus, it has lead the various vendors to try to build the comprehensive solutions, namely Hybrid Transactional/Analytical Processing (HTAP) systems(the term HTAP was defined by Gartner). Particularly, SAP HANA[9] has the engines that are optimized for OLAP workloads, at the same time it also supports ACID transactions. As for HyPer[10], which primarily was using row-wise processing to support both types of workloads, now it presented the opportunity to use a columnar format to run analytical requests in a more efficient manner. The following examples such as Peloton[11], OctopusDB[1] and  SnappyData[16] also belong to HTAP systems.  
One of the solutions that most radically departs from existing architectures was proposed by Jens Dittrich and Alekh Jindal - a new type of database system, coined OctopusDB[1]. By dynamically mimicking several types of workloads(OLAP, OLTP, Hybrid of OLAP and OLTP and etc.), OctopusDB shows a considerably better performance. Moreover, depending on the use-case it may also emulate data stream management systems by replacing some of the stores with streaming windows. \\ Another important goal of HTAP systems(aside from maintaining different system components compatible in order to create an illusion of a single system) is to support OLAP queries for analysis over the real-time data. The fact that HTAP systems might reduce the need of transforming data from one system to another(via ETL), seems like a good step towards the support of real-time data. We believe that the exploration of the techniques related to more interactive queries, can contribute to the real-time characteristics of HTAP systems. Among the techniques that can handle more interactive queries Approximate Query Processing(AQP) has recently gained a substantial attention. By processing the compact synopsis of the data rather than the whole dataset, the methods for AQP are often the only viable solution to provide interactive response times when exploring massive datasets and handling high speed data streams[12]. Thus, we particularly want to ascertain the role that AQP can play in the architecture of an HTAP system, for facilitating real-time queries on the latest data. %%Meanwhile, it became evident that a system performance can be further enhanced by applying techniques from completely different perspectives. 
Furthermore, when it is plausible for a system to retrieve results approximately rather than exactly, AQP could be a reasonably good fit to further improve the performance of query processing (especially, OLAP queries). It could also enhance HTAP by answering the OLAP queries over new data that even has not been included yet and when it is guaranteed that a given amount of incoming data will not change the error of an estimation. Several successful examples(e.g. BlinkDB, SnappyData on Facebook's Presto) have already proved that there are benefits to be gained by integrating approximation features into existing DBMSs[5]. So we believe that combining OctopusDB and AQP techniques could be feasibly good solution for improving HTAP \textit{freshness} and the performance of our system. 
Thus, our goal is to provide a user a tool called Blinktopus, that will allow  to build his own prototype of OctopusDB with embedded AQP techniques.

In Section 2 we start with the brief discussion of what is the idea behind OctopusDB and AQP along with the concepts of AQP main data synopses. We present our contributions as follows:
\begin{itemize}
\vspace{0.05 cm}
\item{In Section 3 we propose a novel concept of a new system called Blinktopus and explain why exactly one type of AQP data synopsis was chosen for our system.}
\item{In Section 4 we introduce the experimental part of our project. Here we provide an evaluation on the benefits of Blinktopus's functionality based on the results of the tests performed on the Storage Views(SVs) with the different physical layouts, namely LogSV, ColSV and RowSV.}
\end{itemize}
In Section 5 we present the related work. We conclude this paper in Section 6 and propose future directions for our research in Section 7.

\section{Fundamentals}
In this section we discuss the core idea of OctopusDB, its motivation and  architecture. Further we explain the main concept of AQP and elaborate AQP main data synopses such as samples, histograms, sketches, wavelets and etc.
\subsection{OctopusDB}
As it was mentioned earlier, most companies currently have to manage how to keep the \textit{zoo} of systems, each designed for a specific use-case scenarios. Even though some enterprises that invest a lot might be able to afford the maintenance of different systems and/or by integrating they might eventually get a single all-embracing solution. Whereas it might be still beyond the strength of most small- and medium-sized companies. Moreover, systems that adhere the principles of traditional DBMSs might not be able to deliver the best performance or to fulfill some use-cases. Thus, the motivation to create new database system with \textit{one-size-fits-all} architecture which is suitable for different scenarios is quite eloquent. 
OctopusDB is one of the representatives of above-mentioned type of system. The most striking difference of OctopusDB's architecture is that in comparison with other DBMSs it does not have a predefined fixed store. All data is stored in the central log named \textit{primary log}. The data is being collected into the log through the creation of logical log-entries of the insert/update operations, each record is identified by internally unique log sequence number \textit{lsn}. OctopusDB exploits the write-ahead loging(WAL) protocol and stores the primary log on durable storage(HDD or SSD). Depending on the workload, a so-called Storage View (SV) can represent the entire central log or some of its parts in different physical layouts(e.g., row-store, column-store, streaming windows etc.). For instance, for transactional queries OctopusDB can create Row SV, Col SV - for OLAP queries. Moreover,  it can optionally decide to create other materialized view such as Index SV or even to mimic streaming systems. At the same time, primary log is another SV for OctopusDB.  However, here remains a non-trivial optimization problem of \textit{storage view selection}(i.e., to determine automatically which type of SV is proper to be created). The component of Octopus DB called \textit{Holistic Storage View Optimizer} copes with this issue. It maintains all SVs including primary log as well as it is responsible for creation and maintenance of secondary SVs. By means of scanning the indices and whole tables, later  depending on cost model the optimizer resolves either to create new SV or to maintain an existing one. Furthermore, by applying transformation cost model it might transform different SVs one into another. Additionally, it can remove from system all SVs. Thus, based on the workload and arbitrary occurring use-cases the holistic SV optimizer operates a \textit{storage view lattice}, within the Storage View Store of the OctopusDB.

\subsection{Approximate Query Processing}
Main Idea. Synopses for Massive Data: Histograms. Sketches.
More on histograms. 
Probably something on HLL, if it's eventually included in Blinktopus.

\section{Blinktopus}
In this section we introduce a novel concept of our system called Blinktopus and give an explanation for the chosen type AQP data synopsis implemented in Blinktopus.  
\subsection{Architecture}

\section{Experimental Part}
Evaluation of the test results:
 
1. Comparison of runtimes of different types of SVs.
2. Comparison of the diff SVs by means of percentage representation.

\section{Related Work}
Some related work has to be mentioned here.

\section{Conclusions}
Sum up what happened

\section{Future Work}


\section*{Acknowledgement}
Thanks to Gabriel(DBSE). Also we would like to thank Jindal, Dittrich and Mozafari for their awesome ideas that really inspired this project.

\begin{thebibliography}{1}

\bibitem{octopus_architecture}
Dittrich, Jens, and Alekh Jindal. "Towards a One Size Fits All Database Architecture." CIDR. 2011.
\bibitem{octopusdb}
Jindal, Alekh. "OctopusDB: flexible and scalable storage management for arbitrary database engines." (2012).
\bibitem{mimicking_octopus}
Jindal, Alekh. "The mimicking octopus: Towards a one-size-fits-all database architecture." VLDB PhD Workshop. 2010.
\bibitem{aqp_engines}
Mozafari, Barzan. "Approximate query engines: Commercial challenges and research opportunities." SIGMOD, 2017.
\bibitem{aqp_handbook}
Mozafari, Barzan, and Ning Niu. "A Handbook for Building an Approximate Query Engine." IEEE Data Eng.
Bull. 38, no. 3 (2015): 3-29.
\bibitem{one_size_does_not_fit_all}
M. Stonebraker and U. Cetintemel. "One Size Fits All": An Idea Whose Time Has Come and Gone. In ICDE, pages 2–11, 2005.
\bibitem{hekaton}
C. Diaconu, C. Freedman, E. Ismert, P.-˚A. Larson, P. Mittal, R. Stonecipher, N. Verma, and M. Zwilling. Hekaton: SQL Server’s memory-optimized OLTP engine. In SIGMOD, pages 1243–1254, 2013.
\bibitem{monetdb}
P. Boncz, M. Zukowski, and N. Nes. MonetDB/X100: Hyper-Pipelining Query Execution. In CIDR, 2005.
\bibitem{saphana}
F. Färber, N. May, W. Lehner, P. Große, I. Müller, H. Rauhe, and J. Dees. The SAP HANA Database – An Architecture Overview. IEEE DEBull 35(1):28–33, 2012.
\bibitem{hyper}
A. Kemper and T. Neumann. HyPer – A Hybrid OLTP'\&'OLAP Main Memory Database System Based on Virtual Memory Snapshots. In ICDE, pages 195–206, 2011.
\bibitem{peloton}
A. Pavlo, J. Arulraj, L. Ma, P. Menon, T. C. Mowry, M. Perron, A. Tomasic, D. V. Aken, Z. Wang, and T. Zhang. Self-Driving Database Management Systems. In CIDR, 2017.
\bibitem{cormode}
Cormode, Graham, Minos Garofalakis, Peter J. Haas, and Chris Jermaine. "Synopses for massive data: Samples, histograms, wavelets, sketches." Foundations and Trends in Databases 4, no. 1–3 (2012): 1-294.
\bibitem{c_store}
Mike Stonebraker, Daniel Abadi, Adam Batkin, Xuedong Chen, Mitch Cherniack, Miguel Ferreira, Edmond Lau, Amerson Lin, Sam Madden, Elizabeth O'Neil, Pat O'Neil, Alex Rasin, Nga Tran and Stan Zdonik. "C-Store: A Column-oriented DBMS." VLDB, pages 553-564, 2005. 
\bibitem{h_store}
H-Store. http://hstore.cs.brown.edu/
\bibitem{memsql}
MemSQL. http://www.memsql.com/
\bibitem{snappy}
SnappyData. https://www.snappydata.io/
\end{thebibliography}

\end{document}